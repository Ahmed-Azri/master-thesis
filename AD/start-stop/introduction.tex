% Introduction file

\section{Introdução}

Uma técnica comum em comunicação assíncronas é a utilização do modelo 
\emph{Start-Stop}. 
Esse modelo é muito simples e não dispõe de mecanismos sofisticados para 
recuperação de falhas, sincronização e estabilização. 
Auto establização é um conceito que garante que um determinado sistema 
seja capaz de se estabilizar sem intevenções externas. 

O presente trabalho propõe uma abordagem auto estabilizante para o 
mecanismo de comunicação assíncrona \emph{Start-Stop}. 
A abordagem apresentada é baseada no seguinte cenário: Considere uma 
transmissão de $A$ para $B$. 
Em um determinado momento da transmissão $B$ começa sua recepção após
$A$ ter iniciado seu envio de dados fazendo com que a transmissão seja
comprometida.
Para tal, como garantir que $B$ sempre se recupere de uma transmissão 
mal feita e continue a receber os dados corretamente?
