\section{Auto Estabilização (self estabilization)}

Um algorítmo distribuído é auto estabilizante se, partindo de um estado 
inicial, ele alcança estados permitidos em um número finito de de passos
\citep{erciyes2013distributed}.
Algorítmos auto estabilizantes recuperam-se de falhas. 
Uma vez recurados, eles se mantém assim. 
Tipicamente, esses algorítmos rodam em \emph{background} e nunca terminam 
sua execução.

Um estado permitido é um estado cujo o sistema tem total controle sobre 
as propriedades desejadas pela aplicação e é livre de erros.
Sistemas auto estabilizantes recuperam-se de falhas sem interversões externas.
Os algorítmos de auto estabilização são tipicamente implementados como 
máquinas de estado. 
Cada nó do sistema transita entre os estados baseado 
nas entradas fornecidas ao programa e no estado atual em que ele se encontra.
O estado global da aplicação é representado pela união de todos os estados
dos nós da aplicação distribuída.

Duas propriedades são essenciais para os sistemas auto estabilizantes. 
Fechamento (\emph{closure}) e Convergência (\emph{convergence}).
O fechamento é uma propriedade de garante que o sistemas faz apenas transições
válidas entre quaisquer dois estados.
A propriedade de Convergência define que o sistema sempre transita entre 
estados permitidos (legais) partindo de qualquer outro estado.
