% Arquivo com a sessão de Solução

\section{Solução}

Para solucionar o problema do presente trabalho, pode-se 
levar em consideração o período de intervalo de transmissão \emph{rest}.
Ao finalizar uma recepção $B$ sempre espera por um bit \emph{stop}. 
Considerando que transmissões seriais possuem \emph{bound-rate} e que 
tanto transmissor quanto receptor processam a transmissão utilizando
seu \emph{clock} interno, pode-se ajustar o tempo de espera 
\emph{rest} para o tempo de duração de uma transmissão que parte de $A$ 
para $B$. 

Essa abordagem torna a transmissão dos \emph{frames} mais lenta, logo 
a trasmissão global também é afetada. 
Contudo, ela garante que $B$ pode falhar em receber algum \emph{frame}, 
mas em seu próximo recebimento $B$ se recupera de falha na trasmissão 
não comprometendo as transmissões que virão.
Conforme o modelo teórico de um sistema auto estabilizante, temos que uma 
máquina de estados finitos para o modelo de comunicação \emph{Start-Stop}
transita entre 3 estados ${start, data, stop}$. 
A solução apresentada adiciona o estado \emph{rest}. 
Assim teríamos o conjunto de estados definido por ${start, data, stop, rest}$.

Como toda transmissão sempre começa com o bit 0, então $B$ por mais que
tenha perdido a transmissão corrente, irá aguardar o período \emph{rest}
e em seu próximo recebimento não terá comprometido as demais transmissões.
As propriedades de fechamento (closure) e convergência são satisfeitas 
através da solução proposta. 
