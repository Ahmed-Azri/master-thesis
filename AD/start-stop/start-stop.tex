% Start Stop Transmission file

\section{Transmissão Start-Stop}

Transmissões Start-Stop são tipicamente utilizadas para comunicações seriais
assíncronas. 
Existem alguns padrões como o RS-232 
\footnote{RS-232 foi uma padronização para comunicação de equipamentos da 
década de 60}
para esse tipo de comunicação utilizada por protocolos como UART
\footnote{UART é um protocolo de Recepção/transmissão assíncrona universal
que traduz dados entre formas paralelas e seriais}.

Basicamente, transmissões assíncronas começam com o envio de um bit de 
\emph{start}. 
O bit 0 (o bit lógico \emph{LOW}) representa o início da 
comunicação. 
Em seguida são esperados 8 bits (1 byte) de dados. 
O byte de dados são tipicamente codificados no formato de caracteres ASCII.
Ao final o transmissor envia um bit de \emph{stop} representado por 
1 (o bit lógico \emph{HIGH}).
Cada transmissão concluída consiste no envio de um \emph{frame} quadro. 
Conforme citado em \citep{asynchronous2009manish}, alguns equipamentos 
possuem um intervalo para recebimento do próximo \emph{frame} chamado 
\emph{rest}.
