\chapter{Conclusão e trabalhos futuros}

O presente trabalho apresentou o uso de grafos no contexto de SDN.
Um grafo da rede em tempo real vai de encontro com uma das principais 
vantagens em se desacoplar o plano de controle do plano de dados que é obter 
uma visão global da rede.
O sistema proposto mantém um estado fiel, consistente e dinâmico da rede real, 
facilitando a tarefa de gerenciamento de um rede definida por software e 
reduzindo o volume de computação nos nós da rede.
Grafos são uma modelagem direta e precisa da topologia de uma rede. 
Eles deveriam ser um recurso básico, uma premissa em controladores SDN para 
representar a rede.

A abstração em grafos foi identificada em outros cenários de rede como 
\emph{cloud computing}. 
\emph{OpenStack}, um dos mais populares sistemas para gerenciamento de 
ambientes virtualizados, possui uma visualização da topologia da rede
em um de seus módulos Neutron~\citep{openstacksite}.
Essa abstração foi combinada em uma solução SDN para implementar ambientes 
isolados com redes \emph{multi-tenant} \citep{lcn2013}.
Seria interessante avaliar uma possível combinação unificada com uma visão
compartilhada da rede.

Como uma proposta futura, pode-se criar um visualizador em tempo real do grafo 
que interaja com o administrador da rede e mostre, de uma maneira simples, 
toda a operação da rede.

Assim como o controlador POX, o módulo proposto, ao ter seu processo terminado,
não persiste as informações de estado da rede.
Todo o grafo computado é perdido.
Em função disso, um banco de dados em grafos, distribuído, poderia ser 
utilizado para persistir o grafo, e as informações do estado da rede, de 
maneira confiável e tolerante a falhas. 

Algoritmos genéricos em grafos poderiam ser implementados como uma biblioteca
para o módulo.
Assim, estabelecida uma periodicidade, esses algoritmos poderiam ser computados
no grafo e seus resultados publicados como extensão da API.

Uma avaliação comparando algoritmos de roteamento como OSPF e BGP com uma 
solução que faz utilização do grafo seria interessante a fim de mostrar a 
redução na computação dos nós da rede.
