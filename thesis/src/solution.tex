\chapter{Solução}

A solução proposta por este trabalho foi implementada em cima do controlador
POX \citep{pox}. 
Uma abstração da rede em forma de grafo foi construída como um módulo 
do controlador da rede.
Esse grafo, em tempo real, representando a rede é a base para toda computação
executada sobre a rede.
A solução objetiva simplificar o gerenciamento em redes reduzindo o volume 
de computações ao longo dos nós da rede aproveitando a separação do plano 
de dados e do plano de controle.

\subsection{A abstração em grafos}

O grafo propost é representado por $G=(V, A)$, em que $V$ e $A$ são conjuntos
finitos de vértices e arestas, respectivamente.
Cada vértice $v \in V$ representa um computador (\emph{host}) ou comutador
(\emph{switch}) dentro da rede.
Cada aresta $u \to v \in A$ representa um enlace (\emph{link}) entre dois
vértices.
O peso das arestas $g(u, v)$ descreve a quantidade de \emph{bytes} trafegados
na aresta entre os dois vértices.

\subsection{Controlador}

O controlador POX é um arcabouço para elaboração de módulos/programas 
em Redes Definidas por \emph{Software}.
Totalmente voltado para pesquisa, o POX é um controlador simples, 
escrito na linguagem de programação \emph{Python}.

Ao ser carregado, o POX, executa o seu núcleo (\emph{core}). 
Esse módulo principal é responsável por carregar os demais 
módulo e garantir a comunição entre eles.

Ele exporta uma interface baseada em eventos. 
Eventos que descrevem ou representam ações ocorridas na rede.
Um módulo implementado dentro do controlador POX pode ser um 
produtor/consumidor de eventos.


\section{Projeto de implementação}

