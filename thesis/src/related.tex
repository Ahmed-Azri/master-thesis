\chapter{Trabalhos relacionados}

Alguns trabalhos apresentam ou sugerem uma abordagem em grafos como uma 
solução em Redes Definidas por Software.
Essa seção apresenta os trabalhos relacionados ao presente projeto de 
dissertação e discute suas características.

\section{A abordagem em grafos}

A abordagem de representar a rede na forma de um grafo foi mencionada 
por Casado \emph{et al.} em um dos primeiros artigos sobre SDN
\citep{martin2010virtualizing}.
No entanto, nenhum detalhe de implementação é apresentado.
Em um trabalho futuro, uma solução SDN foi desenvolvida através de 
diferentes topologias de rede dentro do contexto de \emph{datacenter} 
em que a abstração em grafos não foi adotada \citep{ripcord}. 

Raghavendra \emph{et al.} apresenta um módulo em grafos com capacidade 
de atualização dinâmica com uma API para algorítmos em grafos
\citep{ramya2012dynamic}.
Esse trabalho não possui nenhuma integração com algum controlador SDN,
que é a base da avaliação do presente trabalho.

O controlador \emph{Onix} \citep{teemu2010onix} foi projetado em torno do 
conceito NIB (\emph{Network Information Base}), que é uma base 
de informações da rede.
Essa base mantém uma visão global da rede de maneira similar à 
MIB (\emph{Management Information Base}) implementada sobre o
protocolo SNMP.
Essa representação baseada em grafos é alcançada indexando cada
entrada de elemento em relação a seus vizinhos.
