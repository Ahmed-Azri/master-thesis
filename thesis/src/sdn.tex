\section{Redes definidas por software (SDN)}


\subsection{Definição}
SDN é um modelo de implementação de redes em que separa-se o plano de controle
(decisões) do plano de encaminhamento (comutação de pacotes). 
Através desse modelo o OpenFlow \citep{nick2008openflow} é uma proposta SDN 
bem sucedida. 
Essa separação dos planos de controle e de dados torna o funcionamento da
rede mais flexível.

As motivações das redes definidas por software vem desde o surgimento das 
redes \emph{overlay} \citep{clark2006overlay}, às quais as redes são definidas
sobre outras redes, como camadas isoladas.
O paradigma de Redes definidas por software abre a possibilidade de se 
desenvolver novas aplicações que controlem os elementos de comutação de uma
rede física de maneiras impensadas no passado \citep{guedes2012redes}.

As aplicações modernas são muito dinâmicas. 
Quando se trata de um ambiente distribuído, por exemplo em um datacenter, 
essas aplicações demandam que o funcionamento de sua rede seja também 
dinâmico e flexível às necessidades das aplicaçãos. 
SDN permite satisfazer essas condições de funcionamento. 

Novos protocolos, servições e redes podem ser facilmente desenvolvidos através
do paradigma das Redes definidas por software.
Ao tornar os elementos de comutação de uma rede programáveis, a 
experimentação torna-se mais direta e independente dos fabricantes de 
\emph{hardware} e equipamentos de rede. 
Isso permite que pesquisadores façam experimentações e gerem inovações 
na área de Redes de computadores \citep{nick2008openflow}.

A Internet é uma rede de redes de computadores madura. 
No entanto possui muitas deficiências. 
Para resolver o problema da evolução da Internet, tornar os equipamentos 
programáveis é uma boa aboradagem. 
Muitas redes \emph{overlay} como \emph{PlanetLab} 
\citep{peterson2006experiences} e \emph{GENI} \citep{berman2014geni} tentaram
resolver o problema mas a necessidade de alteração nos equipamentos de rede
fizeram com que tivessem baixa aceitação.
Foi nesse ponto que a solução OpenFlow em SDN acertou.


\subsection{Plano de dados}

\subsection{Plano de controle}


\subsection{Características}
