\section{Redes definidas por software (SDN)}


\subsection{Definição}
SDN é um modelo de implementação de redes em que separa-se o plano de controle
(decisões) do plano de encaminhamento (comutação de pacotes). 
O protocolo OpenFlow \citep{nick2008openflow} é uma proposta para se 
criar soluções em SDN. 
Essa separação dos planos de controle e de dados torna o funcionamento da
rede mais flexível.

As motivações das redes definidas por software vem desde o surgimento das 
redes \emph{overlay} \citep{clark2006overlay}, às quais as redes são definidas
sobre outras redes, como camadas isoladas.
O paradigma de Redes definidas por software abre a possibilidade de se 
desenvolver novas aplicações que controlem os elementos de comutação de uma
rede física de maneiras impensadas no passado \citep{guedes2012redes}.

As aplicações modernas são muito dinâmicas. 
Quando se trata de um ambiente distribuído, por exemplo em um datacenter, 
essas aplicações demandam que o funcionamento de sua rede seja também 
dinâmico e flexível às necessidades das aplicações. 
SDN permite satisfazer essas condições de funcionamento. 

Novos protocolos, servições e redes podem ser facilmente desenvolvidos através
do paradigma das Redes definidas por software.
Ao tornar os elementos de comutação de uma rede programáveis, a 
experimentação torna-se mais direta e independente dos fabricantes de 
\emph{hardware} e equipamentos de rede. 
Isso permite que pesquisadores façam experimentações e gerem inovações 
na área de Redes de computadores \citep{nick2008openflow}.

A Internet é uma rede de redes de computadores madura. 
No entanto possui muitas deficiências. 
Para resolver o problema da evolução da Internet, tornar os equipamentos 
programáveis é uma boa abordagem.
Muitas redes \emph{overlay} como \emph{PlanetLab} 
\citep{peterson2006experiences} e \emph{GENI} \citep{berman2014geni} tentaram
resolver o problema mas a necessidade de alteração nos equipamentos de rede
fizeram com que tivessem baixa aceitação.
Foi nesse ponto que a solução OpenFlow em SDN acertou.


\subsection{Plano de dados}

O plano de dados é responsável pelo encaminhamento de pacotes. 
Esse encaminhamento pode ser implementado através de \emph{hardware} 
comum em roteadores e comutadores. 
O encaminhamento de pacotes consiste em executar algumas operações 
como alterar cabeçalhos dos pacotes, descartá-los e encaminhar para alguma 
porta específica do equipamento.

\subsection{Plano de controle}

O plano de controle consiste em tomar as decisões de como as operações do 
plano de dados serão executadas.
Como uma entidade separada do plano de dados, o plano de controle, para 
tomar as decisões, precisa ter uma visão topológica e global da rede. 
A visão global pode levar a entender uma entidade centralizada. 
No entanto o plano de controle pode ser distribuído. 
Decisões como roteamento, \emph{firewall}, priorização de pacotes são 
responsabilidade do plano de controle.
O plano de controle tem uma natureza de sistemas de tempo real.

\subsection{Características}

As Redes definidas por software tornam a rede programável.
Essa característica dá flexibilidade na administração da rede.
O plano de controle isolado, cria uma entidade logicamente centralizada.
A evolução das redes em relação às aplicações é simplificada. 
Novos experimentos em redes podem ser criados da mesma forma como se cria 
novos algoritmos e aplicações.
Isso torna essa evolução menos custosa, tanto tecnicamente 
quanto financeiramente.
SDN é apenas um modelo, um novo paradigma em redes de computadores.
