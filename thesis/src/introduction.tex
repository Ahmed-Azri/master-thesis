\chapter{Introdução}
\label{chap:introduction}


\section{Motivação}

Redes definidas por software (SDN) separam o plano de dados do plano 
de controle \citep{guedes2012redes}.
Os ambientes de programação projetados para prover aplicações em SDN são 
chamados de controladores SDN.
Eles são conhecidos também como sistemas operacionais de rede, pois criam uma
camada que isola o controle do acesso físico dos elementos de rede através 
de uma interface padronizada.

Aplicações em rede executam algoritmos em grafos.
Em muitos casos essa computação é feita em diferentes nós da rede de maneira 
repetitiva.
Em função da natureza logicamente centralizada do plano de controle é possível
minimizar a quantidade de aplicações computando as mesmas informações.

\section{Problema}

Em sua maioria, as aplicações em Redes Definidas por \emph{Software} necessitam
de uma visão topológica da rede.
Uma visão global é um dos principais aspectos do paradígma 
\citep{martin2010virtualizing}.
Grafos são uma modelagem direta para representar de maneira natural e precisa 
a topologia de uma rede.
Em função disso, grafos deveriam ser um recurso básico, uma premissa em 
controladores SDN para representar a rede.

A representação em grafos pode ser útil para módulos internos de um controlador
ou até para serviços/aplicações externos que dependam de informações 
topológicas ou do estado da rede.

\section{Proposta}

O presente trabalho apresenta uma abstração da rede na forma de um grafo 
para o gerenciamento de redes no plano controle possibilitando automatizar 
detecção de falhas e provisionamento para um grafo dinamicamente atualizado.
Uma implementação em um sistema utilizando OpenFlow \citep{nick2008openflow}
e sua avaliação experimental são apresentados como prova de conceito.

Um módulo dentro do controlador SDN armazena o grafo que representa diretamente
a rede.
Algoritmos em grafos podem ser computados uma única vez e seus resultados 
armazenados para consultas posteriores por outros módulos.

A interação de outros módulos com o grafo pode ser encapsulada e 
semanticamente bem definida em uma interface padronizada.
Essa implementação pode ser adaptada para diferentes sistemas, de maneira que
o módulo pode utilizar de recursos como memória local, banco de dados remoto,
ser distribuído, ter controle de concorrência, paralelismo e outras 
características relevantes para um determinado cenário de utilização.

\section{Soluções}

A ideia de visão topológica da rede está presente em vários controladores SDN.
O controlador NOX \citep{gude2008nox} trabalha a topologia da rede baseando-se
em eventos.
Um banco de dados distribuído é proposto no controlador Onix 
\citep{teemu2010onix}.
Em \citep{hinrichs2009pratical} um sistema com regras de predição 
estabelecem essa abstração.
Linguagens de domínio específico (DSL) como o Frenetic 
\citep{foster2011frenetic} e o Pyretic \citep{monsanto2013composing} permitem
a recuperação de informações topológicas da rede.
Em \citep{ramya2012dynamic} uma API em grafos é apresentado dentro do contexto
de computação na nuvem.
Nossa proposta apresenta a avaliação da abordagem em grafos e sua implementação
no controlador POX \citep{pox2015}, de código aberto e voltado para pesquisa.

\section{Divisão do documento}

O presente projeto de dissertação, primeiramente, apresenta uma fundamentação
teórica sobre as Redes Definidas por \emph{Software}, o protocolo OpenFlow e
as possíveis arquiteturas de implementação.
Um capítulo sobre os trabalhos/pesquisas relacionados comparam e descrevem
soluções em relação à abordagem em grafos.
Em seguida, todo o projeto de implementação e como a solução foi planejada
é apresentado. 
Decisões de projeto como a adaptação do controlador e como o grafo foi 
modelado.

O ambiente de experimentos (\emph{testbed}) e simulação é apresentado no 
capítulo \ref{cap:experiments}.
Em seguida, são apresentados os resultados das análises executadas sobre os
experimentos.
Uma avaliação da solução como um todo.

Ao final são descritos os trabalhos futuros do projeto de dissertação e uma
conclusão sobre os resultados obtidos ao longo do trabalho.
