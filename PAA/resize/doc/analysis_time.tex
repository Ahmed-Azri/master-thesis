\subsection{Comparação entre tempo de execução e ordem de complexidade}

Para a análise de tempos de execução e complexidade, 
foi elaborado um programa que calcula,
para cada algoritmo,
a complexidade ($O$),
o tempo de execução ($t$) em milissegundos e
a razão de complexidade por tempo ($r = \frac{O}{t}$).
para os dois algoritmos.
Para comparação, 
partido do pressuposto que os tempos da solução em grafo é maior,
o programa apresenta a comparação em forma de
razão das complexidades ($R_O = \frac{O_g}{O_d}$),
razão dos tempos de execução ($R_t = \frac{t_g}{t_d}$) e
razão (proporção) entre complexidade e tempo ($R_r = \frac{r_g}{r_p}$).

Foram analisadas as imagens disponibilizadas para o trabalho e
o resultado segue no apêndice ``A''.

Considerando as análises de complexidade \ref{O-g} e \ref{O-d},
a razão das complexidades é 
$R_O = \frac{O(w^2hlog_2(wh))}{O(wh)} = O(wlog_2(wh))$.
Esta razão deveria corresponder à razão entre tempos ($R_t$).
Entretanto,  
na prática $R_t < R_O$.{\footnote{
A imagem ``small.ppm'', de dimensão $(5,3)$, 
é a excessão para todas as análises,
pois sua execução foi mais rápida que a precisão do relógio do sistema.
}.
Para explicar essa observação,
é importente notar que a complexidade do algoritmo de 
programação dinâmica (PD) é fixa.
Ou seja, independente do caso, ela se mantém a mesma.
Já a complexidade do algoritmo de \emph{Dijkstra} 
varia de acordo com o caso, 
acomodando diferentes análises de complexidade,
tanto amortizada quanto de caso médio ou mínimo.
O encerramento antecipado da busca
após atingir o menor caminho já apurado,
também causa uma redução no seu tempo de execução.
Além disso, a otimização de código realizada pelo compilador\footnote{
Opção ``-O3'' do GCC.
},
complicam ainda mais a relação entre complexidade e tempo de execução.

Mesmo assim, os tempos em grafo foram consideravelmente 
maiores que os em PD;
mais de 373 vezes maiores na média.
Isto comprova a maior ``rapidez'' da versão em PD,
prevista nas análises de complexidade.

Comprovando o comportamento estável do algoritmo de PD,
observa-se que a razão de complexidade por tempo ($r$) se mantém estável,
diferente da versão em grafo.

O tamanho da imagem ($w \times h$)
é um fator previsível e confirmado de influência no aumento do tempo de execução. 
Um outro fator observado de redução dos tempos do algoritmo em grafos é
a quantidade de caminhos de baixa energia, 
ou quantidade de pixels ``escuros''.
Neste caso, é possível considerar um aumento na probabilidade de
encerramento antecipado. 
Ou seja, quanto mais caminhos de energia ``nula'',
melhor é o desempenho do programa em grafos.

