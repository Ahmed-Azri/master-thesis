\subsection{Correção dos algoritmos e programas de menor caminho}

Um módulo específico de testes foi programado
para comparar os resultados parciais 
ao longo do processo de redimensionamento.
O teste verifica, após a execução de cada algoritmo,
(1) a igualdade dos caminho mínimos,
(2) a igualdade do somatório das energias do caminho mínimo e
(3) a igualdade das imagens usada em ambos os algoritmos.

Vários problemas, principalmente na análise de resultados,
foram encontrados e os testes foram
se tornando mais abrangentes e precisos ao longo dos experimentos.
Um exemplo de análise que mudou os testes foi a observação que,
após uma divergência de escolha de caminho mínimo de mesma energia total,
ocorriam, em sequência, várias outras divergências de resultados.
O motivo é que as imagens se tornavam diferentes.
Então, encontrada uma escolha de caminhos diferentes 
de mesma enegia mínima,
ao final das verificações, 
a imagem usada em um dos algoritmos é copiada para a do outro.
No caso de eventual divergência inválida, 
o programa de teste relata detalhes para análise manual, 
mas não iguala as imagens.

Após experimentos em várias imagens,
não foram encontradas nenhuma divergência inválida.
Praticamente todas as divergências de escolha de caminho mínimo
ocorreram em áreas ``escuras'', nas quais a enegia total é zero.
Além disso, o programa de otimização baseado em grafo
percorria a imagem sentido inverso ao programa de programação dinâmica.
Após ajustar o sentido, tornando ambos ascendentes, 
o número de divergências diminuiu muito na média.