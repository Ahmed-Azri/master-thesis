
\subsection{Solução em progração dinâmica}
\label{sol:dp}

Considere-se o problema original e a função objetiva $f(I)$,
já apresentada e implementada em ``\nameref{sol:graph}''. 
A questão atual é: 
existe forma de expressar matematicamente $f(I)$?

Considere-se $e(x,y)$ a energia e
$f(x,y)$ a função que retorna a energia acumulada do pixel $I(x,y)$
ao longo do caminho de menor energia.
Perceba-se que, se a imagem só possuir uma linha ($h = 1$), 
a solução $s$ é encontrar o pixel com menor energia na coluna.

\begin{gather}
f(x,1) = e(x,1) \nonumber\\
s = min(\{f(i,1)\}_{i=1}^w)
\label{one}
\end{gather}

Se a imagem possuir duas linhas, 
a solução é encontrar, na segunda linha, 
o pixel com menor energia acumulada,
ou seja, sua energia mais a menor energia dentre 
os pixels adjacentes superiores na primeira linha:

\begin{gather}
f(x,2) = e(x, 2) + min(e(x-1, 1), e(x, 1), e(x+1, 1)) \nonumber\\
s = min(\{f(i,2)\}_{i=1}^w) 
\label{specific}
\end{gather}

Generalizando-se para uma imagem de altura $h$, 
a solução é encontrar, na última linha, 
o pixel com menor energia acumulada.
Partindo da equação \ref{one} como passo base e 
da equação \ref{specific} como indução:

\begin{gather}
f(x,y)=\left\{ 
  \begin{array}{l}
    e(x,y),\quad \text{se $y = 1$}\\
    e(x,y)+min(f(x-1,y-1),f(x,y-1),f(x+1,y-1))
  \end{array} \right. \nonumber\\
s = min(\{f(i,h)\}_{i=1}^w)
\label{general-f}
\end{gather}

Note-se que o caminho completo (solução completa)
não pode ser obtido partindo-se apenas do pixel $s$.
Por isso, 
suponha-se uma matriz $S$ com as mesmas dimensões da imagem.
Para cada pixel, 
$S[x,y]$ representa um vetor de deslocamento 
para um pixel adjacente superior.
Como o deslocamento sempre ocorre 
na direção e sentido da solução do problema, 
os valores de $S$ precisam apenas indicar deslocamentos
em direção perpendicular, no caso, no eixo $x$.
Ou seja, $S[x,y] = \Delta{x}$.

\[ S[x,y] = \left\{ 
  \begin{array}{l}
    -1, \quad \text{se $f(x-1, y-1)$ é mínimo em relação à I(x,y).}\\
     0, \quad \text{se $f(x, y-1)$ é mínimo em relação à I(x,y) ou $y = 1$.}\\
     1, \quad \text{se $f(x+1, y-1)$ é mínimo em relação à I(x,y).}\\
  \end{array} \right.\]
\label{general-S}

Note-se que 
(1) a energia é sempre positiva, 
(2) o caminho de menor energia ocorre apenas entre pixels visinhos e 
(3) em uma única direção e sentido.
Considere-se um pixel $p_n$, 
pertencente ao conjunto ordenado da solução $S$.
Independente de qualquer algoritmo,
o subconjunto $\{p_1,..,p_n\}$ representa 
o menor caminho do pixel inicial $p_1$ até $p_n$\footnote{
Em \cite[pg. 383]{clrs} existe uma prova por contradição para 
a sub-estrutura ótima do problema de menor caminho
em grafo com caracteristicas compatíveis 
ao do problema de menor energia tratado no presente trabalho.
}. 
Ou seja, a cada passo da execução do algoritmo, 
o menor caminho até o momento é encontrado.
Todos estes fatores, aliado à prova por indução apresentada,
indicam a existência de uma sub-estrutura ótima 
para o problema do caminho mínimo.

Além da sub-estrutura ótima, 
a função$f$ possui sobreposição ({\it overlapping} de sub-problemas,
pois para cada pixel $I(x,y)$, ela supõe a execução recursiva para 
todos os pixels superiores atingíveis.
Essa é uma oportunidade para ``memorização'' ({\it memoization})
dos resultados parciais. 
Considere-se $M[x,y]$ como uma matriz das dimensões da imagem,
na qual cada valor representa a energia acumulada em um pixel.
Nesse caso, a ordem de construção de $M$ 
se torna fundamental: ela deve ser descentente, ou seja,
na mesma direção mas de sentido inverso ao problema.
Nesse caso, para $y$ variando de $1$ até $h$, nessa ordem:

\begin{gather}
M[x,y]=\left\{ 
  \begin{array}{l}
    e(x,y),\quad \text{se $y = 1$}\\
    e(x,y)+min(M[x-1,y-1],M[x,y-1],M[x+1,y-1])
  \end{array} \right. \nonumber\\
s = min(\{M(i,h)\}_{i=1}^w)
\label{general-M}
\end{gather}

O algoritmo para essa soluçãoi é composto das seguintes etapas:

\begin{itemize}
\item Construir as matrizes $M$ e $S$ em $O(wh)$ passos.
\item Encontrar o pixel $s$ com a menor energia acumulada 
    na última linha em $O(w)$ verificações.
\item Definir, baseando-se em $S$, 
    o caminho mínimo partindo de $s$ e 
    retirá-lo da imagem com $O(h)$ iterações.
\end{itemize}

Considerando-se o método de análise de complexidade 
adotado anteriormente:

\begin{equation}
\label{O-d}
\begin{tabular}{r l}
$T(w,h,\Delta{w})$ & $ = \sum \limits_{i=w}^{w - \Delta{w}}{O(ih) + O(i) + O(h)}$ \\
$T(w,h)$ & $= O(wh) + O(w) + O(h)$ \\ 
         & $= O(wh)$ \\
$T(n)$   & $= O(n^2)$ \\
\end{tabular}
\end{equation}

O espaço de memória necessário inclui a imagem, as matrizes $M$ e $S$,
cada uma na ordem de $O(wh)$,
o vetor de caminho mínimo, da ordem de $O(h)$ e algumas variáveis de 
controle, na ordem de $O(1)$.
Como os dados se somam, a complexidade de espaço é 
$max(3 \times O(wh), O(h), O(1)) = O(wh) = O(n^2)$.

Comparando-se os dois algoritmos,
a solução do {\it Seam Carving} baseada em programação dinâmica é
muito mais ``rápida'' que a solução baseada em grafo. 
Ambas possuem a mesma complexidade de espaço.




