%\chapter{Soluções Propostas}
\section{Soluções Propostas}

\subsection{Solução adotada}
\label{sol:intro}

O problema de redimensionamento em função do conteúdo da imagem 
pode ser solucionado de diversas maneiras. O presente trabalho 
baseia-se em um algorítmo chamado \emph{Seam Carving}\cite{shai2007seam}. 
Seu algorítmo modela a imagem como uma matriz de pixels, 
sendo que cada pixel é uma cor em formato RGB. 
A matriz representando uma imagem é dada na forma 
$I(w \times h)$, onde $w$ é a largura e $h$ é a altura da imagem. 

O algorítmo percorre caminhos conectados dessa matriz. 
Um caminho conectado é uma sequência de pixels na imagem em uma 
coluna ou uma linha. 
Para manter a proprção retangular da imagem,
deve-se remover uma linha ou uma coluna inteira por vez. 
No contexto, será considerada apenas a remoção de colunas.
Para a remoção de linhas basta utilizar a imagem transposta,
fazendo, ao final, uma nova transposição para retornar a orientação
original da imagem.

A energia representa a ``importância'' dos pixels. 
Para escolher o conjunto de pixels,
busca-se uma sequência de pixels conectados, caminho, com menor energia.
Um pixel é ``importante'' se ele é diferente de seus vizinhos. 
A energia de um caminho é o somatório das energias
(importâncias) dos pixels ao longo do caminho. 
Os caminhos com menor energia são os escolhidos para serem removidos. 
Esta técnica de redimensionamento busca 
preservar o conteúdo relevante da imagem.

O presente trabalho apresenta duas formas de soluções para o problema:
uma solução baseada em \emph{grafos} e uma em \emph{programação dinâmica}. 
Para a abordagem em \emph{grafos}, foi utilizado o algorítimo de 
\emph{Dijkstra} para encontrar o caminho de menor energia. 
A solução em \emph{programação dinâmica} foi baseada no artigo
que descreve o \emph{Seam Carving}\citep{shai2007seam}.

\subsection{Função da energia}

É possível adotar diferentes funções para estabelecer a energia
de cada pixel.
Aliás, várias funções de energia posem ser aplicadas em uma mesma imagem.
Note-se que pixels não destacados, 
que se misturam com os demais ao seu redor,
são menos ``importantes''.
A questão é: como decidir se um pixel é ou não destacado dos demais?

Considere-se cada pixel $(x,y)$ de uma imagem $\mathbf{I}$ como um número.
Suponha-se os conjuntos dos pontos dispostos 
nas linhas $\mathbf{I}_x$ e colunas $\mathbf{I}_y$
como uma amostragem de valores de uma função contínua.
Nesse caso, a derivada parcial em cada amostra de  $\mathbf{I}$,
$x \in \mathbf{I}_x$ e
$y \in \mathbf{I}_y$,
indica o ângulo da reta tangente ao ponto:
$\hat{x} = \frac{\partial}{\partial x}\mathbf{I}$ e
$\hat{y} = \frac{\partial}{\partial y}\mathbf{I}$.

\begin{equation}
\label{gradient}
\nabla f(x,y) = \frac{\partial f}{\partial x}\mathbf{I} + 
                \frac{\partial f}{\partial y}\mathbf{I}
\end{equation}

Estes ângulos definem um vetor de duas dimensões 
$\vec{v}(\hat{x},\hat{y})$ que ``aponta'' para a direção 
da mudança de cor em cada pixel da image.
A equação \ref{gradient} representa as variações de direção na
intensidade de cor, ou seja, o {\it gradianete} da imagem.
Quanto maior o valor do gradiente em um ponto, 
mais ``diferente'' ele é de seus visinhos (diferença cromática).
Como existem duas direções (dimensões) na imagem, 
a ``diferença'' total em um pixel $(x,y)$ pode ser obtida com a soma 
dos módulos dos ângulos em cada direção,
que é equivalente à $|\vec{v}|$.
Então, uma função de enrgia que busque destacar diferenças entre
pixels visinhos pode ser expressa na forma geral:

\begin{equation}
\label{energy}
e(\mathbf{I}) = 
    |\frac{\partial}{\partial x}\mathbf{I}| + 
    |\frac{\partial}{\partial y}\mathbf{I}|
\end{equation}

Duas questões permanecem: 
Como definir um número relevante para cada pixel; sua intensidade luminosa?
Como calcular as derivadas parcias em ambas as direções?

Os pixels são compostos por valores de intensidade 
das três cores básicas (RGB).
Para se definir um número único, 
é necessário considerar que o ser humano tem sensibilidade diferente 
para cada cor básica. 
Então, deve-se considerar a sensibilidade visual para 
as cores vermelha, verde e azul:

\begin{equation}
\label{luminosity}
    IL(R,G,B) = 0.30 \times R + 0.59 \times G + 0.11 \times B
\end{equation}

Aplicando-se esta função em todos os pixels, 
obtem-se as intensidades luminosas da imagem $\mathbf{I}$.
A imagem resultante é uma versão em ``escala de cinza''.

O operador de Sobel é uma aproximação imprecisa do gradiente da imagem,
mas é computacionalmente ``barato'' e simples, 
com bons resultados práticos. 
Aplicado em imagem, ele é um filtro que destaca as bordas dos objetos,
intensificando a diferença de um pixel com os demais.
Para cada pixel $(x,y)$, aplica-se uma média ponderada da luminosidade dos
pixels visinhos. 
Como é uma aproximação do gradiente, 
o filtro utiliza duas matrizes, uma para cada direção (horizontal e vertical).

Considere-se $A$ como uma matriz formada pela vizinhança de um pixel.
Sejam $M_x$ e $M_y$ as matrizes das médias ponderadas de cada direção.
Os gradientes aproximados são $G_x = M_x \times A$ e $G_y = M_y \times A$.
Para a média ponderada correta, o somatório dos pesos da matriz $M$ do
operador de Sobel deve ser zero.
Considerando a soma dos módulos dos gradientes:

\begin{equation}
\label{sobel}
    G = \sqrt{(G_x)^2 + (G_y)^2}
\end{equation}

A equação \ref{sobel} representa uma função de energia viável para a
técnica de redimensionamento por \emph{Seam Carving}\citep{shai2007seam}.

Para aplicar o operador de Sobel nos pixels 
das extremidades (bordas) da imagem, 
é necessário definir um valor ``default'' de seus visinhos inexistentes.
É possível considerar a intensidade luminosa como ``mais escura'' (zero).
Todavia, seguindo-se a orientação do trabalho,
utilizou-se a repetição das linhas próximas na forma similar a um ``espelho''.

Considere-se que a multiplicação das matrizes é da forma 
$G'_{(l,l)} = M'_{(l,m)} \times A_{(m,l)}$,
sendo $l$ o número de linhas e $m$ o número de colunas da matriz $M$.
Suponha-se o valor da posição $M_{(i,j)}$, que irá multiplicar 
pelo valor da posição $(j, i)$ da matriz $A$.
Partindo do pixel $(x, y)$, centro da matriz $A$, 
é necessário deslocar 
$\Delta{x} = x - \lfloor\frac{m}{2}\rfloor$ e 
$\Delta{y} = y - \lfloor\frac{l}{2}\rfloor$ 
para encontrar cada pixel correspondente à posição.
Em outras palavras, a posição $(j, i)$ correspondete na imagem $I_{(c_x, c_y)}$ é 
$c_x = \Delta{x} + i$ e $c_y = \Delta{y} + j$.
Perceba-se que, como se trata de multiplicação de matrizes, 
as dimensões da imagem em $A$ são inversas às dimensões de cada matriz $G$.


Considere-se uma imagem $I$ de dimensão máxima $(w, h)$, 
sendo $0 \le x < w$ e $0 \le y < h$.
Considere-se $n_x$ e $n_y$ como o número de colunas e linhas a serem espelhas:
$0 \le n_x \le \lfloor\frac{m}{2}\rfloor$ e 
$0 \le n_y \le \lfloor\frac{l}{2}\rfloor$.


\begin{itemize}
\item 
Para o espelhamento das colunas mais à esquerda 
($x \le \lfloor\frac{m}{2}\rfloor$) e linhas ao topo 
($y \le \lfloor\frac{l}{2}\rfloor$), 
basta-se observar que se 
$(c_x = \Delta{x} + i) < 0 $ e/ou 
$(c_y = \Delta{y} + j) < 0$.
Ou seja,
$c_x = -n_x$ e seu espelho deve ser $c_x = |n_x|$, assim como
$c_y = -n_y$ e seu espelho deve ser $c_y = |n_y|$:
em qualquer dos casos, mesmo sem espelhamento, a possição correspondente é 
$\mathbf{c_x = |\Delta{x} + i|}$ e/ou
$\mathbf{c_y = |\Delta{y} + j|}$.

\item
Nas colunas mais à direita 
($x \ge w - \lfloor\frac{m}{2}\rfloor$) e linhas ao fundo 
($y \ge h - \lfloor\frac{l}{2}\rfloor$), 
basta-se observar que 
$(c_x = \Delta{x} + i) \ge w$ e/ou 
$(c_y = \Delta{y} + j) \ge h$.
Como $x \le (w - 1)$ e $y \le (h - 1)$ (partindo do zero),
$c_x = (w - 1) + n_x$ e seu espelho deve ser $c_x = w - n_x$, assim como
$c_y = (h - 1) + n_y$ e seu espelho deve ser $c_y = h - n_y$.
Note-se que se $p = (q - 1) + r$ e $p' = q - r$, 
partindo de $p$ obtém-se 
$p' = 2q - \mathbf{p} - 1 = 2q - ((q - 1) + r) - 1 = 2q - q + 1 - r - 1 = q - r$.
Ou seja, é possível encontrar $p' = c_{\{x,y\}}$ 
sem se conhecer $r = n_{\{x,y\}}$.
Dai, se $c_x \ge w$, então $\mathbf{c_x = 2w - c_x - 1}$.
Analogamente, se $c_y \ge h$, então $\mathbf{c_y = 2h - c_y - 1}$.

\end{itemize}



\subsection{Solução em grafos}
\label{sol:graph}

O problema de encontrar a costura ({\it seam}) de menor energia de uma imagem
é similar ao problema de encontrar o menor caminho entre 
os pixels adjacentes em uma direção (vertical ou horizontal).
Este é o objetivo da função $f(I)$: dado uma imagem $I$, retornar 
o caminho de menor energia em direção específica.
No caso, será tratado apenas soluções na direção vertical, 
ao longo da dimensão $y$.
Mudar a direção é trivial, como já tratado em ``\nameref{sol:intro}''.
Um caminho vertical pode ter dois sentidos: 
(1) para baixo (descendente ou {\it top-down}) ou
(2) para cima (ascendente ou {\it bottom-up}).
Mesmo que a solução seja a mesma, independente do sentido,
considere-se o sentido ascendente.

Considere-se um grafo direcionado $G=(V,A)$ em que $V$ é o conjunto de vértices 
e $A$ o conjunto de arestas na forma $\{u, v \in V, w \in A | w=(u \to v)\}$.
No caso, $w$ é uma aresta que liga o vértice $u$ ao $v$. 

No modelo adotado, cada vértice $v$ representa um pixel, ou seja $v \equiv (x, y)$. 
As arestas representam as adjacências na vizinhança ascendente do vértice.
Para um pixel qualquer em $I(x, y)$, 
os seus pixels adjacentes superiores são, no máximo:

\begin{equation}
\label{adj_sup}
adj_{\uparrow}(x, y) = \{I(x-1, y-1), I(x, y-1), I(x+1, y-1)\}
\end{equation}

O grafo é representado por uma lista (vetor) de vértices. 
Um vértice, por sua vez, contém os dados do pixel que ele representa e sua 
lista de adjacência (arestas que o liga aos outros vértices).
O peso (distância) das arestas são as energias de cada pixel adjacente.
Assim, a distância inicial não deve ser zero,
mas deve ser a energia do pixel inicial da busca do menor caminho. 

O programa elaborado aloca memória fixa para 4 vértices adjacentes:
3 vértices no máximo e 1 flag de finalização.
O final do vetor de adjacência é marcado pelo valor (flag) ``{\it NONE}'',
que indica o fim de seu conjunto $adj_{\uparrow}(v)$.

O algoritmo adotado para a solução do problema em grafo 
é baseado no clássico \emph{Dijkstra}.
Sua versão original tem complexidade de tempo na ordem de $O(|V|^2)$. 
Como $|V|$ é o número de vértices, para uma imagem com $(w \times h)$ pixels, 
a complexidade correpondente é $O((wh)^2)$.  

Em um grafo não direcionado,
algoritmo de \emph{Dijkstra} parte de um vértice qualquer (dado) e 
percorra todos os demais vértices de um grafo conectado.
Nesse processo, o algoritmo vai ajustando o caminho mínimo ao longo da execução. 
São utilizados dois conjuntos fundamentais:

\begin{itemize}
\item $Q$, que é um mapa (valor-chave $\rightarrow$ vértice), 
    sendo que o valor representa a menor distância acumulada. 
    Inicialmente, $Q$ possui todos os vértices com valores infinitos 
    e o vértice inicial com valor mínimo.
\item $S$, que é um vetor indexado pelo vértice que informa 
    seu sucessor no caminho encontrado, que será mínimo ao final da execução.
    O caminho mínimo pode ser obtido percorrendo-se $S$ do vértice final
    até o vértice inicial.
\end{itemize}

O algoritmo visita todos os vértices em $Q$, 
buscando sempre o de menor distancia ($key$ mínimo).
Como $Q$ retorna sempre o vértice com menor distância,
se as distãncias (pesos) forem sempre positivas,
então um vérice já visitado não precisa ser reconsiderado,
pois qualquer outro caminho até ele será maior ou igual ao já encontrado.
Após visitado o vértice $v$, de menor distancia $d_v$, ele é removido de $Q$.
Para cada um dos seus vértices adjacentes ($u \in adj_{\uparrow}(v)$), 
calcula-se a distancia de $v$ até ele:
$d_u = d_v + d_{((v \rightarrow u))}$. 
Relembrando, $d_{((v \rightarrow u))}$ é 
a distãncia,  ``tamanho'' ou peso da aresta que,
no caso, representa a energia do pixel.
Se a distancia calculada $d_u$ for menor que o atual valor ($key(u) \in Q$),
o valor de sua chave em $Q$ é decrementado para $d_u$ ({\it Decrease-Key}) e 
seu antecessor é definido como o vértice que está sendo visitado: $S[u] = v$.
O algoritmo termina quando $Q$ for vazio. 
O caminho mínimo está em $S$, partindo-se do último vértice visitado.

O escopo original do problema de \emph{Dijkstra} é similar,
mas ligeiramente diferente do {\it Seam Carving}.
Por exemplo, o grafo do modelo adotado é direcionado, porém não possui ciclos.
A condição de parada pode ser mudada para 
o momento que a execução atingir um vértice pertencente 
à primeira linha da imagem\footnote{
Note-se que a linha ou a coluna, inicial e final, dependem da
direção e sentido adotado na execução do programa.
No caso, foi estabelecida a direção vertical e direção ascendente.
}.
Além disso, existem variações do algoritmo, 
como a que foi implementada no presente trabalho, 
que não são baseadas em descremento de chave em $Q$ 
({\it Dijkstra-NoDec})\cite{chen07priorityqueues}.

A solução completa para o problema de encontrar e remover
o caminho de menor enegia da imagem 
deve executar os seguintes passos:

\begin{itemize} 
\item Construção do grafo em $O(wh)$.
\item Busca do menor entre os menores caminhos;
      cada busca em $O((wh)^2)$ partindo de
      cada coluna,
      totalizando $O(w) \times O((wh)^2)$ iterações. 
\item Encontrado o menor caminho, 
      deve-se retirá-lo da imagem em $O(h)$.
\end{itemize}

Para o redimensionamento, 
é necessário considerar a retirada de 
$\Delta{w}$ colunas e/ou $\Delta{h}$ linhas.
A cada coluna ou linha removida, 
o número de pixels considerados é reduzido, 
reduzindo-se gradativamente 
a complexidade para encontrar o próximo caminho de menor energia.

\begin{tabular}{r l}
$T(w,h,\Delta{w})$ & $ = \sum \limits_{i=w}^{w - \Delta{w}}{O(ih) + (O(i) \times O((ih)^2)) + O(h)}$ \\
\end{tabular}

Para fim de simplicidade e comparação, 
considere-se apenas a retirada de uma coluna ou uma linha.

\begin{tabular}{r l}
$T(w,h)$ & $= O(wh) + O(w) \times O((wh)^2) + O(h)$ \\ 
         & $= O(wh) + O(w^3h^2) + O(h)$ \\
         & $= O(w^3h^2)$ \\
\end{tabular}

Dado que $w$ e $h$ são variáveis independentes que tendem a ter valores próximos,
para que seja mantida a forma retangular de uma imagem,
é possível considerar $O(w) = O(h) = O(n)$.
Ou seja:

\begin{tabular}{r l}
$T(n)$ & $= O(n^5)$ \\
\end{tabular}

Para tentar reduzir a complexidade, 
foi elaborado uma versão do \emph{Dijkstra} que utiliza 
uma pilha de prioridade $Q$ ({\it priority queue})
\urldef\webpriq\url{http://rosettacode.org/wiki/Priority_queue#C}
\footnote{
O programa de lista de prioridade usado no trabalho 
foi adaptado do programa disponívle em 
{\webpriq}.
},
mantida internamente ordenada,
com tempo de busca do menor valor em $O(1)$, de inserção  e remoção em $O(log_2{|V|})$.
Com a pilha de prioridade, 
a busca do menor caminho
é $O(|V| \times 2log_2{|V|} + |V|) = O(|V|log_2(|V|))$. 
Considerando-se o método de análise proposto anteriormente:

\begin{equation}
\label{O-g}
\begin{tabular}{>{$}r<{$} >{$}l<{$}}
T(w,h,\Delta{w}) &= \sum \limits_{i=w}^{w - \Delta{w}}{O(ih) + (O(i) \times O((ih)log_2(wh))) + O(h)} \\
T(w,h) & = O(wh) + O(w) \times O((wh)log_2(wh)) + O(h)) \\ 
       & = O(wh) + O(w^2hlog_2(wh)) + O(h) \\
       & = O(w^2hlog_2(wh))\\
T(n)   & = O(n^3log_2(n^2)) = O(n^32log_2(n)) = O(n^3log_2(n))\\
\end{tabular}
\end{equation}

Para tentar redizir ainda mais o esforço computacional,
adotou-se um controle do menor caminho já calculado para colunas anteriores.
Ao longo da busca, 
se o caminho encontrado for maior que o menor já encontrado,
o programa encerra a busca na coluna atual e passa para a próxima, se houver.

Considerando o espaço de memória necessário:

\begin{itemize}
\item imagem com $O(|V|)$ elementos (pixels),
\item grafo com as matrizes de adjacência ocupa $O(k_1+k_2|V|) = O(|V|)$,
    no qual $k_1$ representa o número de variáveis de controle e 
    $k_2$ o tamanho fixo do vetor de vértices adjacentes 
    mais um ponteiro de um pixel correspondente na imagem.
\item três vetores de controle: 
vértices prévios no menor caminho, 
vértices visitados e distância mínima, cada um com $O(|V|)$ elementos,
\item pilha de prioridade, 
    também com $O(|V|)$ de memória alocada antecipadamente.
\end{itemize}

Como todas as estruturas são $O(|V|)$, 
a complexidade de espaço é $O(k|V|) = O(|V|) = O(wh) = O(n^2)$.
Assim como a complexidade de tempo,
o consumo de memória se reduz ao longo da execução,
pois $w$ se reduz para $w - \Delta{w}$.


\subsection{Solução em progração dinâmica}
\label{sol:dp}

Considere-se o problema original e a função objetiva $f(I)$,
já apresentada e implementada em ``\nameref{sol:graph}''. 
A questão atual é: 
existe forma de expressar matematicamente $f(I)$?

Considere-se $e(x,y)$ a energia e
$f(x,y)$ a função que retorna a energia acumulada do pixel $I(x,y)$
ao longo do caminho de menor energia.
Perceba-se que, se a imagem só possuir uma linha ($h = 1$), 
a solução $s$ é encontrar o pixel com menor energia na coluna.

\begin{gather}
f(x,1) = e(x,1) \nonumber\\
s = min(\{f(i,1)\}_{i=1}^w)
\label{one}
\end{gather}

Se a imagem possuir duas linhas, 
a solução é encontrar, na segunda linha, 
o pixel com menor energia acumulada,
ou seja, sua energia mais a menor energia dentre 
os pixels adjacentes superiores na primeira linha:

\begin{gather}
f(x,2) = e(x, 2) + min(e(x-1, 1), e(x, 1), e(x+1, 1)) \nonumber\\
s = min(\{f(i,2)\}_{i=1}^w) 
\label{specific}
\end{gather}

Generalizando-se para uma imagem de altura $h$, 
a solução é encontrar, na última linha, 
o pixel com menor energia acumulada.
Partindo da equação \ref{one} como passo base e 
da equação \ref{specific} como indução:

\begin{gather}
f(x,y)=\left\{ 
  \begin{array}{l}
    e(x,y),\quad \text{se $y = 1$}\\
    e(x,y)+min(f(x-1,y-1),f(x,y-1),f(x+1,y-1))
  \end{array} \right. \nonumber\\
s = min(\{f(i,h)\}_{i=1}^w)
\label{general-f}
\end{gather}

Note-se que o caminho completo (solução completa)
não pode ser obtido partindo-se apenas do pixel $s$.
Por isso, 
suponha-se uma matriz $S$ com as mesmas dimensões da imagem.
Para cada pixel, 
$S[x,y]$ representa um vetor de deslocamento 
para um pixel adjacente superior.
Como o deslocamento sempre ocorre 
na direção e sentido da solução do problema, 
os valores de $S$ precisam apenas indicar deslocamentos
em direção perpendicular, no caso, no eixo $x$.
Ou seja, $S[x,y] = \Delta{x}$.

\[ S[x,y] = \left\{ 
  \begin{array}{l}
    -1, \quad \text{se $f(x-1, y-1)$ é mínimo em relação à I(x,y).}\\
     0, \quad \text{se $f(x, y-1)$ é mínimo em relação à I(x,y) ou $y = 1$.}\\
     1, \quad \text{se $f(x+1, y-1)$ é mínimo em relação à I(x,y).}\\
  \end{array} \right.\]
\label{general-S}

Note-se que 
(1) a energia é sempre positiva, 
(2) o caminho de menor energia ocorre apenas entre pixels visinhos e 
(3) em uma única direção e sentido.
Considere-se um pixel $p_n$, 
pertencente ao conjunto ordenado da solução $S$.
Independente de qualquer algoritmo,
o subconjunto $\{p_1,..,p_n\}$ representa 
o menor caminho do pixel inicial $p_1$ até $p_n$\footnote{
Em \cite[pg. 383]{clrs} existe uma prova por contradição para 
a sub-estrutura ótima do problema de menor caminho
em grafo com caracteristicas compatíveis 
ao do problema de menor energia tratado no presente trabalho.
}. 
Ou seja, a cada passo da execução do algoritmo, 
o menor caminho até o momento é encontrado.
Todos estes fatores, aliado à prova por indução apresentada,
indicam a existência de uma sub-estrutura ótima 
para o problema do caminho mínimo.

Além da sub-estrutura ótima, 
a função$f$ possui sobreposição ({\it overlapping} de sub-problemas,
pois para cada pixel $I(x,y)$, ela supõe a execução recursiva para 
todos os pixels superiores atingíveis.
Essa é uma oportunidade para ``memorização'' ({\it memoization})
dos resultados parciais. 
Considere-se $M[x,y]$ como uma matriz das dimensões da imagem,
na qual cada valor representa a energia acumulada em um pixel.
Nesse caso, a ordem de construção de $M$ 
se torna fundamental: ela deve ser descentente, ou seja,
na mesma direção mas de sentido inverso ao problema.
Nesse caso, para $y$ variando de $1$ até $h$, nessa ordem:

\begin{gather}
M[x,y]=\left\{ 
  \begin{array}{l}
    e(x,y),\quad \text{se $y = 1$}\\
    e(x,y)+min(M[x-1,y-1],M[x,y-1],M[x+1,y-1])
  \end{array} \right. \nonumber\\
s = min(\{M(i,h)\}_{i=1}^w)
\label{general-M}
\end{gather}

O algoritmo para essa soluçãoi é composto das seguintes etapas:

\begin{itemize}
\item Construir as matrizes $M$ e $S$ em $O(wh)$ passos.
\item Encontrar o pixel $s$ com a menor energia acumulada 
    na última linha em $O(w)$ verificações.
\item Definir, baseando-se em $S$, 
    o caminho mínimo partindo de $s$ e 
    retirá-lo da imagem com $O(h)$ iterações.
\end{itemize}

Considerando-se o método de análise de complexidade 
adotado anteriormente:

\begin{equation}
\label{O-d}
\begin{tabular}{r l}
$T(w,h,\Delta{w})$ & $ = \sum \limits_{i=w}^{w - \Delta{w}}{O(ih) + O(i) + O(h)}$ \\
$T(w,h)$ & $= O(wh) + O(w) + O(h)$ \\ 
         & $= O(wh)$ \\
$T(n)$   & $= O(n^2)$ \\
\end{tabular}
\end{equation}

O espaço de memória necessário inclui a imagem, as matrizes $M$ e $S$,
cada uma na ordem de $O(wh)$,
o vetor de caminho mínimo, da ordem de $O(h)$ e algumas variáveis de 
controle, na ordem de $O(1)$.
Como os dados se somam, a complexidade de espaço é 
$max(3 \times O(wh), O(h), O(1)) = O(wh) = O(n^2)$.

Comparando-se os dois algoritmos,
a solução do {\it Seam Carving} baseada em programação dinâmica é
muito mais ``rápida'' que a solução baseada em grafo. 
Ambas possuem a mesma complexidade de espaço.





