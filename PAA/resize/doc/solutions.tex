\chapter{Soluções Propostas}

\section{Solução adotada}

O problema de redimensionamento em função do conteúdo da imagem 
pode ser solucionado de diversas maneiras. O presente trabalho 
baseia-se em um algorítmo chamado \emph{Seam Carving}. Esse 
algorítmo modela a imagem como uma matriz de pixels, onde cada
pixel é uma cor em formato RGB. A matriz representando uma imagem
é dada na forma $I = (wxh)$, onde $w$ é a largura e $h$ é a altura
da imagem. 

O algorítmo percorre caminhos conectados dessa matriz. 
Um caminho conectado é uma sequência de pixels na imagem em uma 
coluna ou uma linha. O algorítmo busca pelos caminhos com menor
energia. Essa energia é dada pela importância dos pixels que 
compõem o caminho. Um pixel é importante se ele é muito diferente
de seus vizinhos. A energia de um caminho é o somatório das energias
(importâncias) dos pixels ao longo do caminho. Os caminhos com menor
energia são os escolhidos para serem removidos. Isso possibilita 
redimensionar uma imagem preservando seu conteúdo relevante.
