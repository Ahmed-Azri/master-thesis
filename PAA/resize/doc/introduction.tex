%\chapter{Introdução}
\section{Introdução}

Atualmente, existem diversas soluções algorítmicas para lidar com o 
problema de redimensionar imagens. O presente trabalho apresenta
soluções em que o redimensionamento é baseado no conteúdo da imagem.
Utilizar essa estratégia faz com que a qualidade do redimensionamento 
seja, para a maioria dos casos, melhor. 

Os dispositivos modernos de visualização de conteúdo (\emph{notebooks,
computadores pessoais, celulares}) possuem diversos tamanhos de tela (área
de visualização). Essas telas podem ser redimensionadas dinamicamente, 
forçando o conteúdo exibido a ser redimensionado para se adaptar à tela.

O redimensionamento em função do conteúdo da imagem é definido por remover
as partes menos importantes da imagem, podendo assim gerar melhores 
resultados.

