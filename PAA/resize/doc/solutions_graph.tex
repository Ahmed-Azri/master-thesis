\subsection{Solução em grafos}
\label{sol:graph}

O problema de encontrar a costura ({\it seam}) de menor energia de uma imagem
é similar ao problema de encontrar o menor caminho entre 
os pixels adjacentes em uma direção (vertical ou horizontal).
Este é o objetivo da função $f(I)$: dado uma imagem $I$, retornar 
o caminho de menor energia em direção específica.
No caso, será tratado apenas soluções na direção vertical, 
ao longo da dimensão $y$.
Mudar a direção é trivial, como já tratado em ``\nameref{sol:intro}''.
Um caminho vertical pode ter dois sentidos: 
(1) para baixo (descendente ou {\it top-down}) ou
(2) para cima (ascendente ou {\it bottom-up}).
Mesmo que a solução seja a mesma, independente do sentido,
considere-se o sentido ascendente.

Considere-se um grafo direcionado $G=(V,A)$ em que $V$ é o conjunto de vértices 
e $A$ o conjunto de arestas na forma $\{u, v \in V, w \in A | w=(u \to v)\}$.
No caso, $w$ é uma aresta que liga o vértice $u$ ao $v$. 

No modelo adotado, cada vértice $v$ representa um pixel, ou seja $v \equiv (x, y)$. 
As arestas representam as adjacências na vizinhança ascendente do vértice.
Para um pixel qualquer em $I(x, y)$, 
os seus pixels adjacentes superiores são, no máximo:

\begin{equation}
\label{adj_sup}
adj_{\uparrow}(x, y) = \{I(x-1, y-1), I(x, y-1), I(x+1, y-1)\}
\end{equation}

O grafo é representado por uma lista (vetor) de vértices. 
Um vértice, por sua vez, contém os dados do pixel que ele representa e sua 
lista de adjacência (arestas que o liga aos outros vértices).
O peso (distância) das arestas são as energias de cada pixel adjacente.
Assim, a distância inicial não deve ser zero,
mas deve ser a energia do pixel inicial da busca do menor caminho. 

O programa elaborado aloca memória fixa para 4 vértices adjacentes:
3 vértices no máximo e 1 flag de finalização.
O final do vetor de adjacência é marcado pelo valor (flag) ``{\it NONE}'',
que indica o fim de seu conjunto $adj_{\uparrow}(v)$.

O algoritmo adotado para a solução do problema em grafo 
é baseado no clássico \emph{Dijkstra}.
Sua versão original tem complexidade de tempo na ordem de $O(|V|^2)$. 
Como $|V|$ é o número de vértices, para uma imagem com $(w \times h)$ pixels, 
a complexidade correpondente é $O((wh)^2)$.  

Em um grafo não direcionado,
algoritmo de \emph{Dijkstra} parte de um vértice qualquer (dado) e 
percorra todos os demais vértices de um grafo conectado.
Nesse processo, o algoritmo vai ajustando o caminho mínimo ao longo da execução. 
São utilizados dois conjuntos fundamentais:

\begin{itemize}
\item $Q$, que é um mapa (valor-chave $\rightarrow$ vértice), 
    sendo que o valor representa a menor distância acumulada. 
    Inicialmente, $Q$ possui todos os vértices com valores infinitos 
    e o vértice inicial com valor mínimo.
\item $S$, que é um vetor indexado pelo vértice que informa 
    seu sucessor no caminho encontrado, que será mínimo ao final da execução.
    O caminho mínimo pode ser obtido percorrendo-se $S$ do vértice final
    até o vértice inicial.
\end{itemize}

O algoritmo visita todos os vértices em $Q$, 
buscando sempre o de menor distancia ($key$ mínimo).
Como $Q$ retorna sempre o vértice com menor distância,
se as distãncias (pesos) forem sempre positivas,
então um vérice já visitado não precisa ser reconsiderado,
pois qualquer outro caminho até ele será maior ou igual ao já encontrado.
Após visitado o vértice $v$, de menor distancia $d_v$, ele é removido de $Q$.
Para cada um dos seus vértices adjacentes ($u \in adj_{\uparrow}(v)$), 
calcula-se a distancia de $v$ até ele:
$d_u = d_v + d_{((v \rightarrow u))}$. 
Relembrando, $d_{((v \rightarrow u))}$ é 
a distãncia,  ``tamanho'' ou peso da aresta que,
no caso, representa a energia do pixel.
Se a distancia calculada $d_u$ for menor que o atual valor ($key(u) \in Q$),
o valor de sua chave em $Q$ é decrementado para $d_u$ ({\it Decrease-Key}) e 
seu antecessor é definido como o vértice que está sendo visitado: $S[u] = v$.
O algoritmo termina quando $Q$ for vazio. 
O caminho mínimo está em $S$, partindo-se do último vértice visitado.

O escopo original do problema de \emph{Dijkstra} é similar,
mas ligeiramente diferente do {\it Seam Carving}.
Por exemplo, o grafo do modelo adotado é direcionado, porém não possui ciclos.
A condição de parada pode ser mudada para 
o momento que a execução atingir um vértice pertencente 
à primeira linha da imagem\footnote{
Note-se que a linha ou a coluna, inicial e final, dependem da
direção e sentido adotado na execução do programa.
No caso, foi estabelecida a direção vertical e direção ascendente.
}.
Além disso, existem variações do algoritmo, 
como a que foi implementada no presente trabalho, 
que não são baseadas em descremento de chave em $Q$ 
({\it Dijkstra-NoDec})\cite{chen07priorityqueues}.

A solução completa para o problema de encontrar e remover
o caminho de menor enegia da imagem 
deve executar os seguintes passos:

\begin{itemize} 
\item Construção do grafo em $O(wh)$.
\item Busca do menor entre os menores caminhos;
      cada busca em $O((wh)^2)$ partindo de
      cada coluna,
      totalizando $O(w) \times O((wh)^2)$ iterações. 
\item Encontrado o menor caminho, 
      deve-se retirá-lo da imagem em $O(h)$.
\end{itemize}

Para o redimensionamento, 
é necessário considerar a retirada de 
$\Delta{w}$ colunas e/ou $\Delta{h}$ linhas.
A cada coluna ou linha removida, 
o número de pixels considerados é reduzido, 
reduzindo-se gradativamente 
a complexidade para encontrar o próximo caminho de menor energia.

\begin{tabular}{r l}
$T(w,h,\Delta{w})$ & $ = \sum \limits_{i=w}^{w - \Delta{w}}{O(ih) + (O(i) \times O((ih)^2)) + O(h)}$ \\
\end{tabular}

Para fim de simplicidade e comparação, 
considere-se apenas a retirada de uma coluna ou uma linha.

\begin{tabular}{r l}
$T(w,h)$ & $= O(wh) + O(w) \times O((wh)^2) + O(h)$ \\ 
         & $= O(wh) + O(w^3h^2) + O(h)$ \\
         & $= O(w^3h^2)$ \\
\end{tabular}

Dado que $w$ e $h$ são variáveis independentes que tendem a ter valores próximos,
para que seja mantida a forma retangular de uma imagem,
é possível considerar $O(w) = O(h) = O(n)$.
Ou seja:

\begin{tabular}{r l}
$T(n)$ & $= O(n^5)$ \\
\end{tabular}

Para tentar reduzir a complexidade, 
foi elaborado uma versão do \emph{Dijkstra} que utiliza 
uma pilha de prioridade $Q$ ({\it priority queue})
\urldef\webpriq\url{http://rosettacode.org/wiki/Priority_queue#C}
\footnote{
O programa de lista de prioridade usado no trabalho 
foi adaptado do programa disponívle em 
{\webpriq}.
},
mantida internamente ordenada,
com tempo de busca do menor valor em $O(1)$, de inserção  e remoção em $O(log_2{|V|})$.
Com a pilha de prioridade, 
a busca do menor caminho
é $O(|V| \times 2log_2{|V|} + |V|) = O(|V|log_2(|V|))$. 
Considerando-se o método de análise proposto anteriormente:

\begin{equation}
\label{O-g}
\begin{tabular}{>{$}r<{$} >{$}l<{$}}
T(w,h,\Delta{w}) &= \sum \limits_{i=w}^{w - \Delta{w}}{O(ih) + (O(i) \times O((ih)log_2(wh))) + O(h)} \\
T(w,h) & = O(wh) + O(w) \times O((wh)log_2(wh)) + O(h)) \\ 
       & = O(wh) + O(w^2hlog_2(wh)) + O(h) \\
       & = O(w^2hlog_2(wh))\\
T(n)   & = O(n^3log_2(n^2)) = O(n^32log_2(n)) = O(n^3log_2(n))\\
\end{tabular}
\end{equation}

Para tentar redizir ainda mais o esforço computacional,
adotou-se um controle do menor caminho já calculado para colunas anteriores.
Ao longo da busca, 
se o caminho encontrado for maior que o menor já encontrado,
o programa encerra a busca na coluna atual e passa para a próxima, se houver.

Considerando o espaço de memória necessário:

\begin{itemize}
\item imagem com $O(|V|)$ elementos (pixels),
\item grafo com as matrizes de adjacência ocupa $O(k_1+k_2|V|) = O(|V|)$,
    no qual $k_1$ representa o número de variáveis de controle e 
    $k_2$ o tamanho fixo do vetor de vértices adjacentes 
    mais um ponteiro de um pixel correspondente na imagem.
\item três vetores de controle: 
vértices prévios no menor caminho, 
vértices visitados e distância mínima, cada um com $O(|V|)$ elementos,
\item pilha de prioridade, 
    também com $O(|V|)$ de memória alocada antecipadamente.
\end{itemize}

Como todas as estruturas são $O(|V|)$, 
a complexidade de espaço é $O(k|V|) = O(|V|) = O(wh) = O(n^2)$.
Assim como a complexidade de tempo,
o consumo de memória se reduz ao longo da execução,
pois $w$ se reduz para $w - \Delta{w}$.
