\subsection{Análise qualitavia dos resultados}

A técnica de redimensionamento {\it Seam Carving} 
produz ótimos resultados.
Vários bons exemplos estão dispostos no apêndice ``B''\footnote{
Durante a elaboração do trabalho,
foi muito mais difícil encontrar exemplos ruins.
}.

Mas, como é natural na maioria das tecnologias, 
o {\it seam carving} possui limitações e casos típicos de 
resultados ruins. 
Um exemplo é a figura \ref{fig:face}\cite{shai2007seam}.
O redimensionamento na altura distorce as faces.
A solução típica é utilizar algoritmos de 
``detecção de faces''.
Normalmente, esses algoritmos retornam áreas retângulares.
O programa de {\it seam carving} pode 
aumentar a energia dessas áreas.

\resizeExamples{face}{Exemplo de distorção de face.}

Outro exemplo de distorção é o da figura \ref{fig:papa}.

\resizeExamples{papa}{}

Nesse caso, 
a cor interna da cruz é de baixa energia.
Mesmo com os contornos de alta energia do filtro de Sobel,
os menores caminhos passam por dentro do objeto.
Nesse caso, assim como em praticamente todos os demais, 
a solução pode ser um software
capaz de aumentar a energia em áras marcadas pelo usuário 
via interface gráfica.
Outra alternativa mais complexa seria utilizar 
algoritmos de detecção de objetos, formas ou textos (OCR), 
aumentando a energia dos pixels de objetos identificados.
Note-se um exemplo de distorção similar na figura \ref{fig:car}.

\resizeExamples{car}{Exemplo de distorção de formas.}

Os exemplos ruins mais comuns são de imagens condensadas ou
distorçoes de formas e escalas.
O exemplo da figura \ref{fig:ufmg} apresenta uma distorção
de forma de letras.
Note-se que o ``M'' é a forma com mais destaque dentre
todas as letras.

\resizeExamples{ufmg}{Exemplo de distorção de letras.}

Note-se na figura \ref{fig:ufmg} 
a diferênça da distorção com a mudança 
da matriz do operador de Sobel. 

\resizeExamples{ufmg2}{Exemplo de distorção com diferente matriz do filtro de Sobel.}

Imagens nas quais as escalas das formas ou objetos é fundamental,
como mapas, plantas ou esquemas,
tendem a não terem bons resultados com o {\it seam carving}.
Note-se o exemplo da figura \ref{fig:map}.
Nesses casos, 
e em vários outros casos de distorção,
o uso de técnicas convencionais de dimensionamento de imagem ({\it scaling}
podem representar uma melhor alternativa.

\resizeExamples{map}{Exemplo de distorção de escala.}

Outro exemplo destacado no artigo \cite{shai2007seam} é
de imagens condensadas, como a da figura \ref{fig:street}
nas quais o filtro de Sobel destaca muitos objetos.

\resizeExamples{street}{Exemplo de redimensionamento horizontal drástico.}

Em caso de redimensionamento mais radicias ou
mudança da direção do redimensionamento,
o resultado pode mudar consideravelmente.
Note-se a diferença entre a figura \ref{fig:commonwealth1},
na qual foi aplicada redução horizontal de 50\% e 
a mesma imagem original redimensionada em 50\% em ambas as direções na 
figura \ref{fig:over}.

\resizeExamples{commonwealth1}{Exemplo de redimensionamento com bom resultado.}

\resizeExamples{over}{Exemplo de piora na qualidade com a mudança da direção do redimensionamento.}

Mesmo com suas limitações, 
a técnica de {\it seam carving} é extremamente relevante e amplamente adotada.
Além disso, 
ela é flexível e tem grande potencial para se combinar 
com outras técnicas de redimensionamento.
