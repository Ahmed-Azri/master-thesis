\chapter{Introdução}

\paragraph*{}

O presente trabalho objetiva apresentar e analisar soluções para
o problema {\it Longest Common Subsequence (LCS)}. O problema 
consiste em encontrar a maior subsequência comum à duas ou 
mais cadeias de caracteres ({\it strings}), nesse caso em 
específico, apenas duas.

Para compreender o problema é necessário entender a definição
de \emph{ subsequência} no atual contexto:
\begin{verbatim}
    Uma subsequência é um subconjunto dos elementos da sequência (palavra)
    com a ordem estritamente crescente (não necessariamente contíguos)
\end{verbatim}

Segundo o documento que especifica esse trabalho, há uma restrição para os
\emph{segmentos} que compõem uma subsequência. Essa restrição consiste em 
estabelecer um tamanho mínimo para esses segmentos.
\begin{verbatim}
    Um segmento é uma subsequência contígua de uma dada sequência (palavra)
\end{verbatim}

Em função dessas definições, podemos definir o problema como:
\begin{verbatim}
    Dadas duas sequências $X=\{x_1, x_2, ..., x_n\}$ e 
    $Y=\{y_1, y_2, ..., y_m\}$, encontre a maior subsequência 
    $Z=\{z_1, z_2, ..., z_k\}$ comum a $X$ e $Y$
\end{verbatim}
